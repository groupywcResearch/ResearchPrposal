\documentclass[]{report}


\section*{GROUP MEMBERS}
\section*{1. NIWAGABA ROBERT 14/U/13287/EVE }
\section*{2. NAFUNA MERETH GAMBWA 14/U/23611/EVE }
\section*{3. ASIIMIRWE ABRAHAM 14/U/5390/PS }
\section*{4. MAWANDA SAUL 14/U/22411/PS }
\section*{5. OBONYO EMMANUEL 214012922}

% Title Page
\title{HOUSE RENTAL MANAGEMENT SYSTEM}
\author{GROUP COURSE WORK}


\begin{document}
	\maketitle


\tableofcontents 
\chapter{Introduction} 
\section{Background to the Study}
Housing has a central importance to quality of life with considerable economic, social, cultural and personal significance. Though a country’s national prosperity is usually measured in economic terms, increasing wealth is of diminished value unless all can share its benefits and if the growing wealth is not used to redress growing social deficiencies, one of which is housing [1]. Housing plays a huge role in revitalizing economic growth in any country, with shelter being among key indicators of development. 
The universal declaration of human rights gives one of the basic human rights as the right to a decent standard of living, central to which is the access to adequate housing [2]. Housing as a basic human right demands that urban dwellers should have access to a decent housing, defined as one that provides a foundation for rather than being a barrier to good physical and mental health, personal development and fulfillment of life objectives [3]. The focus of this research project is basically managing housing for low income, medium and high incomes households. Most families choose to rent houses based on their income and family situations; unfortunately, there may not be enough good quality rental housing for these families [4]. 

Developing rental houses comes with many advantages especially to the Landlords who are able to increase their profits through rent paid by the tenants. Increased number of tenants and Landlords makes management difficult especially for the landlords who are losing huge sum of money through tenants who evade rent. 
The above statement gives a clear declaration as to why house rental management system need to be developed. Some of the tenants run away without paying the rent.

\section{Statement of the Problem} 
Over the years landlords/property managers have had a problem in maintaining and managing their customers and their own records. Management has become difficult because of issues that include:Data growth Data, Lack of computerized system Currently most landlords/property managers use the manual system in recording and maintaining their property and customers data, Data security is not assured In a manual way, data is recorded on books/papers which may easily get damaged leading to loss of data, there is no database to store information Potential of data loss or damage is very high because data is stored on tangible files. Lack of these crucial requirements makes management of the tenants and houses very difficult as some tenants may end up not paying rent. 


\section{objectives} 
\subsection{General objective}
To develop a house rental management system that allows the user to view customer’s data as well as houses record.


\subsection{Specific Objective}
• To study and analyze the requirement specifications of the house rental management system
• to define the problems that are faced by the current system.
•To design a system that allows the users to add, edit, search and delete data from the database 
•To implement and test the performance of the application

\section{Scope}
The project scope is the clear definition of the boundaries or limits of the investigation. The project focuses on studying and analyzing several literatures concerning the house rental management system until the implementation of the system. 

\section{Significance of the Study / Justification}.
storing and maintaining the tenants data will be easy.
data security will be assured since the system will be automated.

\chapter{literature review}
Literature review is a text written by someone to consider the critical points of current knowledge including substantive findings as well as theoretical and methodological contributions to a particular topic. Main goals are to situate the current study within the body of literature and to provide context for the particular reader (Cooper, 1998).

\section{GOVERNMENT STRATEGY AND INCENTIVES IN THE HOUSING SECTOR} 
Government initiatives in assisting house owners in management have proven to be pathetically slow with many of the houses provided being economically and socially irrelevant, this further prompting the rise of informal settlement (Macoloo, 1994).
\section{THE ROLE OF THE PRIVATE SECTOR IN HOUSE MANAGEMENT}
Private sector housing management is defined as any process which is not connected at all with the actions of the state neither directly constructed by state nor financially sponsored by the state where production is not expected to have a social element[8].
Ambrose and Barlow have argued that three factors are important in influencing the level of new house building. These are direct capital investment by the state for public housing, state support for production and consumption and changes in the profitability of house builders in the private sector[1].
The private sector can play an important role in housing provision provided that the state offers sufficient and appropriate incentives to the sector [6].
The clear motivation that underlies the private sector is profit (or potential profitability) with profit maximizing options being in the context of housing, producing and selling more of the product; reducing the cost of production through lower raw material and wage costs and finally increasing the price of the product or service [6].\
Profitability in housing is advocated to be based on three variables; House prices, land prices and building costs, where:
Profit=House prices-{Land prices + Building costs} [8].

\section{THE ROLE OF RELATIONAL DATABASE MANAGEMENT SYSTEM (RDBMS)}
Database Management System (DBMS) has replaced the file system data management by having a pool of data that can be shared by multiple application programs and users concurrently. DBMS also provide logical and physical data independence, so that changing of data structure or application program will not affect one another.
\chapter{methodology}
\section{Introduction}
The project’s software development processes will be achieved using various techniques. These techniques will involve data collection, data analysis, system design and analysis, system implementation and finally system validation and testing as explained in this section.

\section{Data collection}
Here, I will carry out a study to gain an understanding of the customers (tenants) current system and problems experienced in this system through interviews, observations, and participations. I will use the obtained data to determine the viability of the system being proposed in terms of technical, economic and social feasibility.

\section{Data analysis}
At this stage, I will gather information about what the customer needs and define the problems the system is expected to solve. I will also include customers‟ business context, products functions and its compatibility. I will gather requirement such as software like the programming language to use, database model and hardware needed such as laptop, printers etc  
 
\section{System Design}
At this stage, I will make an overall design of the system architecture and physical design which includes User Interface and Database design. It is at this stage that I will identify any faults before moving onto the next stage. The output of this stage is the design specification which is used in the next stage of implementation. 
\section{system Implementation }
At this stage, I will begin coding as per the design specification(s). The output of this step is one or more product components built according to a pre-defined coding standard and debugged, tested and integrated to satisfy the system architecture requirement. 
\section{system Testing }
At this stage, I will ensure both individual and integrated whole are methodically verified to ensure they are error free and satisfy customer requirement. I will involve both unit testing of individual code module, system testing of the integrated product and acceptance testing conducted by or on behalf of customer. I will ensure bugs found are corrected before moving to the next stage. I will also prepare, review and publish product documentation at this stage. 

\section{Recommendations}
Our project is meant to satisfy the needs of rental house owners. Several user friendly interfaces have also been adopted. This package shall prove to be a powerful in satisfying all the requirements of the users It is with utmost faith that I present this software to you hoping that it will solve your problems and encourage you to continue appreciating technology because it is meant to change and ease all our work that seems to be very difficult. I don‟t mean that my project is the best or that I have used the best technology available it just a simple and a humble venture that is easy to understand. However, I would encourage anyone who has the ability to advance it using advanced technologies so as to increase its capabilities.

\section{Conclusion}
In conclusion, the software can be used as an inventory system to provide a frame work that enables the mangers to make reasonable transactions made within a limited time frame. Each transaction made on the system go hand in hand with the data being updated in the database in our case it is MYQSL which is the back end. Last but not least it is not the work that played the ways to success but ALMIGHTY GOD.





        

\begin{thebibliography}{10}
	\bibitem{Ambrose}Ambrose, P. and Barlow, J. (1987), Housing Provision and House Building in Western Europe: Increasing Expenditure, Declining Output, Housing Markets and Policies under Fiscal Austerity, London, Greenwood Press.
	\bibitem{United}United Nations, (1948), The Bill of Human Rights.
	\bibitem{Seedhouse}Seedhouse, D. (1986), Foundation for Health Achievement, Health Policy, vol. 7, issue, 3.
	\bibitem{Cooper}Cooper , M. (1998), Ideas to develop a literature review, vol. 3, page, 39.
	\bibitem{ Levin}Levin, K. (1999), Database Management Systems: How to use Relational Databases, vol. 2, no 4.
	\bibitem{Peter}Henry Peter Gommans *, George Mwenda Njiru **, Arphaxad Nguka Owange **(2014)International Journal of Scientific and Research Publications, Volume 4, Issue 11, November 2014 1	ISSN 2250-3153 www.ijsrp.org
	\bibitem{Erguden}Erguden, S. (2001), Low cost housing policies and constraints in developing countries, International conference on spatial development for sustainable development, Nairobi.
	
	\bibitem{Golland}Golland, A. (1996), Housing supply, profit and housing production: The case of the United Kingdom, Netherlands and Germany, Journal of Housing and the Built Environment, vol.11, no1.
	
\end{thebibliography}
\appendix

\end{document}   














References
Appendices
–Time Schedule (see example in chapter 2 slides)

\begin{abstract}
\end{abstract}

\end{document}          
